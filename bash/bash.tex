\documentclass[a4paper, twoside, 8pt]{extarticle}
\usepackage[
    left=1.2cm,
    right=1.2cm,
    top=2.25cm,
    bottom=1.25cm]{geometry}
\usepackage{multicol}

\usepackage{fancyhdr} % extensive control of page headers and footers
\makeatletter
\fancypagestyle{mypagestyle}
{\newpage \fancyfoot[C]{} \renewcommand{\footrulewidth}{0pt}}
\makeatother
\pagestyle{mypagestyle}
\headsep 5pt            

\usepackage{minted} % highlighted source code
\usepackage[usenames,dvipsnames]{color} % colour control
\usepackage{xcolor} % driver-independent color extensions
\usepackage{amssymb} % provides an extended symbol collection

\usepackage{enumitem} % control layout of itemize, enumerate, description
\newenvironment{enumx} {
	\begin{enumerate}[leftmargin=*]
	\setlength{\topsep}{0pt}
	\setlength{\itemsep}{0pt}
	\setlength{\parskip}{0pt}
	\setlength{\parsep}{0pt}
	}
{\end{enumerate}}
\newenvironment{itemx} {
	\begin{itemize}[leftmargin=*,noitemsep,topsep=0pt]
	%\setlength{\itemsep}{0pt}
	%\setlength{\parskip}{0pt}
	%$\setlength{\parsep}{0pt}
	}
{\end{itemize}}

\usepackage{parskip} % layout with zero \parindent, non-zero \parskip
\usepackage{titlesec} % selects alternative section titles
\titlespacing\section{0pt}{6pt plus 4pt minus 2pt}{0pt plus 2pt minus 2pt}
\titlespacing\subsection{0pt}{6pt plus 4pt minus 2pt}{0pt plus 2pt minus 2pt}
\titlespacing\subsubsection{0pt}{6pt plus 4pt minus 2pt}{0pt plus 2pt minus 2pt}

\usepackage{lastpage} % reference last page
\usepackage{Alegreya}

\newcommand{\manualbreak}{\vspace*{\fill}\columnbreak}
\newcommand{\cmdblack}{$\blacksquare$}
\newcommand{\cmd}{$\bigstar$}
 
\usepackage[utf8]{inputenc}
\usepackage[T1]{fontenc}

\begin{document}
\renewcommand{\footrulewidth}{0.4pt}
\fancyhead[LE,LO]{Linux/Bash -- notes}
\fancyhead[RO,RE]{date: \today}
\fancyfoot[RF]{author: Remigiusz Suwalski}
\fancyfoot[LF]{page \thepage/\pageref{LastPage}}

\begin{multicols}{3}
\section{Unix utilities and shell builtins}
% Based on https://en.wikipedia.org/wiki/Template:Unix_commands

\subsection{File system}
\begin{enumx}
	\item [\cmd] \textbf{cat} concatenates and prints files:
	\item [\texttt{A}] shows all nonprinting characters,
	\item [\texttt{b}] numbers nonempty output lines,
	\item [\texttt{n}] numbers all output lines,
	\item [\texttt{s}] suppresses repeated empty output lines.
    % tac -r -s 'x|[^x]'
	\item [\cmd] \textbf{tac} does the same in reverse.
	\item [\cmd] \textbf{rev} reverses lines characterwise.
\end{enumx}

\begin{enumx}
	\item [\cmdblack] \textbf{chgrp} changes group ownership.
	
	\item [\cmdblack] \textbf{chmod} changes permissions of a file:
	\item [\texttt{ugoa}] permissions of the owner, group, other/all users,
	\item [\texttt{+-=}] adds, removes or sets selected file mode bits,
	\item [\texttt{rwx}] selects file mode bits: read/write/execute (4/2/1).
	
	\item [\cmdblack] \textbf{chown} changes owner of a file.
	
	\item [\cmd] \textbf{umask} sets file mode creation mask.

	\item [\cmdblack] \textbf{touch} changes file timestamps:
	\item [\texttt{a}] only the access time,
	\item [\texttt{m}] only the modification time,
	\item [\texttt{t}] uses custom stamp instead of current time,
	\item [\texttt{c}] does not create files.
\end{enumx}

\begin{enumx}
	\item [\cmd] \textbf{shasum} prints or checks SHA message digests:
	\item [\texttt{a}] algorithm: 1, 224, 256, 384, 512, 512224 or 512256,
	\item [\texttt{b}] reads in binary mode,
	\item [\texttt{c}] checks SHA sums read from the ,,files''.

	\item [\cmd] See also \textbf{cksum} (CRC checksums) and \textbf{md5sum}.
\end{enumx}

\begin{enumx}
	\item [\cmd] \textbf{dd} converts and copies a file:
	\item [\texttt{if=}] reads from a file instead of standard input,
	\item [\texttt{of=}] writes to a file insteaed of standard output,
	\item [\texttt{bs=}] up to ,,bytes'' bytes at a time,
	\item [\texttt{count=}] copies only ,,n'' input blocks.
\end{enumx}

\begin{enumx}
	\item [\cmdblack] \textbf{cp} copies files and directories:
%	\item [\texttt{a}] never follows symlinks, preserves all attributes,
	\item [\texttt{b}] makes a backup of each existing destination file,
	\item [\texttt{f}] removes an existing destination file if needed,
	\item [\texttt{i}] prompts before overwrite,
	\item [\texttt{n}] does not overwrite existing files,
	\item [\texttt{L}] always follows symlinks in ,,source'',
	\item [\texttt{P}] never follows symlinks in ,,source'',
	\item [\texttt{r}] copies directories recursively,
	\item [\texttt{s}] makes symbolic links instead,
	\item [\texttt{l}] hard links files instead,
	\item [\texttt{t}] copies all ,,source'' arguments into ,,directory'',
	\item [\texttt{T}] treats ,,destination'' as a normal file,
	\item [\texttt{u}] copies only newer source files,
	\item [\texttt{v}] explains what is being done.
	
	\item [\cmdblack] \textbf{mv} moves (renames) files:
	\item [\texttt{b}] makes a backup of each existing destination file,
	\item [\texttt{i}] prompts before overwriting,
	\item [\texttt{f}] does not prompt before overwriting,
	\item [\texttt{n}] does not overwrite existing destination files.
	\item [\texttt{t}] moves all ,,source'' arguments into ,,directory'',
	\item [\texttt{T}] treats ,,destination'' as a normal file,
	\item [\texttt{u}] moves only newer source files,
	\item [\texttt{v}] explains what is being done.
	
	\item [\cmdblack] \textbf{rm} removes files or directories:
	\item [\texttt{f}] never prompts,
	\item [\texttt{i}] always prompts,
	\item [\texttt{r}] removes directories and their contents.

	\item [\cmd] See also \textbf{rmdir} (directories removal).

	\item [\cmd] \textbf{mkdir} makes directories (\texttt{mkdir p}: with parents as needed, no error if existing).
\end{enumx}

\begin{enumx}
	\item [\cmd] \textbf{df} reports file system disk space usage:
	\item [\texttt{h}] prints size in powers of 1024,
	\item [\texttt{i}] list inode information instead of block usage,
	\item [\texttt{t}] limits listing to file systems of given type,
	\item [\texttt{x}] limits listing to file systems not of given type,
	\item [\texttt{T}] prints file systems types.
	
	\item [\cmd] \textbf{du} estimates file space usage:
	\item [\texttt{a}] writes counts for all files, not just directories,
	\item [\texttt{c}] produces a grand total,
	\item [\texttt{d}] the depth at which summing should occur,
	\item [\texttt{h}] prints sizes in human readable format,
	\item [\texttt{s}] diplays only a total.
\end{enumx}


\begin{enumx}
	\item [\cmd] \textbf{file} determines file type.
\end{enumx}

\begin{enumx}
	\item [\cmd] \textbf{fsck} checks and repairs a Linux filesystem:
	\item [\texttt{a}] automatically repairs (without any question!),
	\item [\texttt{t}] specifies the type(s) of filesystem to be checked,
	\item [\texttt{A}] tries to check all filesystems in one run,
	\item [\texttt{M}] skips mounted filesystems,
	\item [\texttt{R}] skips the root filesystem.
\end{enumx}

\begin{enumx}
	\item [\cmd] \textbf{ln} makes hard links between files
	(not directories; only in the same file system):
	\item [\texttt{s}]  makes symbolic links instead.
\end{enumx}

\begin{enumx}
	\item [\cmd] \textbf{ls} lists directory contents:
	\item [\texttt{a}] does not ignore entries starting with dot, 
	\item [\texttt{F}] appends indicator to entries, 
	\item [\texttt{h}] prints human readable sizes, 
	\item [\texttt{i}] prints the index number of each file, 
	\item [\texttt{l}] prints permissions, number of hard links, owner, group, size, last-modified date as well, 
	\item [\texttt{r}] reverses order while sorting,
	\item [\texttt{R}] lists subdirectories recursively, 
	\item [\texttt{S}] sorts by file size (largest first), 
	\item [\texttt{t}] sorts by modification time (newest first), 
	\item [\cmd] \textbf{tree} folds lower case to upper case characters.
\end{enumx}

\begin{enumx}
	\item [\cmd] \textbf{mount} mounts a filesystem.
\end{enumx}

\begin{enumx}
	\item [\cmd] \textbf{pwd} prints name of current directory.
\end{enumx}

\begin{enumx}
	\item [\cmd] \textbf{split} splits a file into pieces:
	\item [\texttt{a}] generates suffixes of length ,,n'' (default 2),
	\item [\texttt{b}] puts ,,size'' bytes per output file,
	\item [\texttt{d}] uses numeric (not alphabetic) suffixes,
	\item [\texttt{l}] puts ,,number'' lines/records per output file,
	\item [\texttt{n}] generates ,,chunks'' output files.
\end{enumx}

\begin{enumx}
	\item [\cmd] \textbf{tar} stores and extracts files from a disk archive:
	\item [\texttt{c}] creates a new archive,
	\item [\texttt{x}] extracts files,
	\item [\texttt{t}] lists the contents of an archive,
	\item [\texttt{v}] verbosely lists files processed,
	\item [\texttt{j}] bzip2 compression,
	\item [\texttt{z}] uses zip/gzip (gz compression),
	\item [\texttt{f}] uses archive file or device (???),
	\item [\texttt{k}] does not replace existing files when extracting.
\end{enumx}

\begin{enumx}
	\item [\cmd] \textbf{tee} (named after the T-splitter used in plumbing) duplicates pipe content:
	\item [\texttt{a}] appends to the given files, does not overwrite,
	\item [\texttt{i}] ignores interrupts.
\end{enumx}

\begin{enumx}
	\item [\cmd] Missing: \textbf{cmp}, \textbf{fuser}, \textbf{pax}, \textbf{type}.
\end{enumx}


\subsection{Processes}
\begin{enumx}
	\item [\cmd] \textbf{chroot} changes the root directory 
	for the current running process and their children.
\end{enumx}

\begin{enumx}
	\item [\cmd] \textbf{at} schedules commands to be executed once, 
	at a particular time in the future: it accepts times of the form 
	\texttt{HH:MM}, \texttt{midnight}, \texttt{noon} or \texttt{teatime}; 
	\texttt{MMDD[CC]YY}, \texttt{MM/DD/[CC]YY}, \texttt{DD.MM.[CC]YY} or 
	\texttt{[CC]YY-MM-DD} (the specification of a date 
	must follow the specification of the time of day).
	You can also give times like \texttt{now + 3 hours}.
\end{enumx}

\begin{enumx}
	\item [\cmd] \textbf{bg} resumes suspended jobs in the background.
	\item [\cmd] \textbf{fg} resumes suspended jobs in the foreground.
	\item [\cmd] \textbf{jobs} lists the active jobs.
	\item [\cmd] \textbf{command \&} runs command in the background.
	% || versus && (error versus success)
\end{enumx}

\begin{enumx}
	\item [\cmd] \textbf{cron}: a daemon executing scheduled commands.
	\item [\cmd] \textbf{crontab} maintain individual users' crontab files.
\end{enumx}

\begin{enumx}
	\item [\cmd] \textbf{kill} sends a \texttt{TERM} signal to a process.
	\item [\cmd] \textbf{killall} kills processes by name.
\end{enumx}

\begin{enumx}
	\item [\cmd] \textbf{ps} reports a snapshot of the current processes.
	\item [\texttt{a}] lifts the ,,only yourself'' restriction,
	\item [\texttt{-e}] selects all processes,
	\item [\texttt{u}] displays user-oriented format,
	\item [\texttt{x}] lifts the ,,must have a tty'' restriction. 
	\item [\cmd] \textbf{pstree} displays a tree of processes.
\end{enumx}

\begin{enumx}
	\item [\cmd] \textbf{nice} changes process priority.
\end{enumx}

\begin{enumx}
	\item [\cmd] \textbf{pgrep}, \textbf{pkill} looks up or signals 
processes based on name and other attributes.
\end{enumx}

\begin{enumx}
	\item [\cmd] \textbf{time} runs programs and summarizes system resource usage. 
\end{enumx}

\begin{enumx}
	\item [\cmd] \textbf{top} displays linux processes.
\end{enumx}


\subsection{User environment}
\begin{enumx}
	\item [\cmd] \textbf{clear} clears the terminal screen.
	\item [\cmd] \textbf{env} runs a program in a modified environment.
	\item [\cmd] \textbf{exit} terminates the calling process.
	\item [\cmd] \textbf{finger} looks up user information.
	\item [\cmd] \textbf{history}  displays the history list. %with line numbers.
%	\item [\cmd] \textbf{logname} prints user's login name.
	\item [\cmd] \textbf{mesg} displays messages from other users.
\end{enumx}

\begin{enumx}
	\item [\cmd] \textbf{passwd} changes user password:
	\item [\texttt{d}] deletes an account's password (makes it empty),
	\item [\texttt{e}] expires an account's password,
	\item [\texttt{n}] sets minimum days to change password,
	\item [\texttt{w}] sets warning days before password expire,
	\item [\texttt{x}] sets the maximum number of days a password remains valid.
\end{enumx}

\begin{enumx}
	\item [\cmd] \textbf{su} changes user ID or becomes superuser.
	\item [\cmd] \textbf{sudo} executes a command as another user.
\end{enumx}

\begin{enumx}
	\item [\cmd] \textbf{uname} prints system information:
	\item [\texttt{a}] all information, in the following order:
	\item [\texttt{s}] the kernel name,
	\item [\texttt{n}] the network node hostname,
	\item [\texttt{r}] the kernel release,
	\item [\texttt{v}] the kernel version,
	\item [\texttt{m}] the machine hardware name,
	\item [\texttt{p}] the processor type,
	\item [\texttt{i}] the hardware platform,
	\item [\texttt{o}] the operating system.
\end{enumx}

\begin{enumx}
	\item [\cmd] \textbf{uptime}: how long has the system been running?
\end{enumx}

\begin{enumx}
	\item [\cmd] \textbf{wall} writes a message to all users,
	\item [\cmd] \textbf{write} sends a message to another user. 
\end{enumx}

\begin{enumx}
	\item [\cmd] \textbf{who} shows who is logged on,
	\item [\cmd] \textbf{w} does the same and shows what they are doing,
	\item [\cmd] \textbf{whoami} prints effective userid.
\end{enumx}



\subsection{Text processing}
\renewcommand\theFancyVerbLine{\normalsize\arabic{FancyVerbLine}}

\begin{enumx}
    \item [\cmd] \textbf{awk} is a pattern scanning / processing language,
    a pseudo-C interpretor.
    Sample code:
\begin{minted}[linenos, numbersep=3pt, frame=lines, framesep=1mm]{bash}
BEGIN {print "- Start -"}
/word/ {print NR ")" $1, $2}
END {print "- End -"}
\end{minted}

\item [] Examples of conditions:
\begin{enumx}
    \item \texttt{/word[0+9]+/}: regular expressions
    \item \texttt{!/word[0+9]+/}: regexes inverted
    \item \texttt{$\sim$} and \texttt{!$\sim$}: matches / does not match.
    \item \texttt{length(\$0) > 18}.
\end{enumx} 

\item [] Important variables:
\begin{enumx}
    \item FS: field separator (tab),
    \item OFS: output field separator,
    \item RS: record separator (new line),
    \item NR: number of the current record,
    \item NF: number of fields in the current record.
\end{enumx} 

\item [\cmd] \textbf{grep} prints lines matching a pattern:
\item [\texttt{c}] prints a count of matching lines instead,
\item [\texttt{e}] uses a ,,regexp'' pattern,
\item [\texttt{f}] obtains patterns from a file,
\item [\texttt{i}] ignores case disctinctions,
\item [\texttt{v}] inverts the sense of matching,
\item [\texttt{w}] selects only lines containing matches that form whole words,
\item [\texttt{n}] prints line numbers as well,
\item [\texttt{A}] prints ,,num'' lines of trailing content,
\item [\texttt{B}] prints ,,num'' lines of leading content,
\item [\texttt{C}] prints ,,num'' lines of both contents,
\item [\texttt{R}] ???,
\item [\cmd] \textbf{sed}: a stream editor filtering/transforming text.
\end{enumx}

\begin{enumx}
	\item [\cmd] \textbf{comm} compares two sorted files line by line.
\end{enumx}

\begin{enumx}
	\item [\cmdblack] \textbf{cut} prints selected parts of lines:
	\item [] \texttt{-}\texttt{-}\texttt{complement} complements the selection,
	\item [\texttt{c}] selects only these characters,
	\item [\texttt{d}] uses ,,delim'' instead of Tab for field delimeter,
	\item [\texttt{f}] selects only these fields,
	\item [\texttt{s}] does not print lines not containing delimeters.
	\item [\cmdblack] \textbf{join} joins lines of two files on a common field.
	\item [\cmdblack] \textbf{paste} merges lines of files.
	\item [\texttt{d}] reuses characters from ,,list'' instead of tabs,
	\item [\texttt{s}] pastes one file at a time, not in parallel.
\end{enumx}

\begin{enumx}
	\item [\cmd] \textbf{diff} compares files line by line:
	\item [\texttt{y}] outputs in two columns,
	\item [\texttt{i}] ignores case differences,
	\item [\texttt{w}] ignores all white space.
	% E Z b B
\end{enumx}

\begin{enumx}
	\item [\cmd] \textbf{fmt} is a simple optimal text formatter, 
	\item [\cmd] \textbf{fold} wraps each line to fit in specified width.
\end{enumx}

\begin{enumx}
	\item [\cmd] \textbf{head}* outputs the first (last) part of files:
	\item [\texttt{c}] the first ,,num'' bytes,
	\item [\texttt{n}] the first ,,num'' lines,
	\item [\cmd] \textbf{tail*} the last ,,num'' bytes:
	\item [\texttt{c}] the last ,,num'' bytes,
	\item [\texttt{n}] the last ,,num'' lines,
	\item [\texttt{f}] outputs appended data as the file grows,
	\item [\texttt{s}] sleeps for approximately ,,n'' seconds between iteration 
\end{enumx}

\begin{enumx}
	\item [\cmd] \textbf{less} is opposite of \textbf{more}.
	\item [\cmd] \textbf{more} is a file perusal filter for crt viewing.
\end{enumx}

\begin{enumx}
	\item [\cmd] \textbf{nl*} numbers lines of files:
	\item [\texttt{s}] adds ,,string'' after line number,
	\item [\texttt{w}] uses ,,number'' columns for line numbers.
\end{enumx}

\begin{enumx}
	\item [\cmd] \textbf{shuf*} generates random permutations:
	\item [\texttt{e}] treats each ,,arg'' as an input line,
	\item [\texttt{i}] treats each number .. through .. as an input line, 
	\item [\texttt{n}] outputs at most ,,count'' lines,
	\item [\texttt{r}] output lines can be repeated (with \texttt{-n}).
\end{enumx}

\begin{enumx}
	\item [\cmdblack] \textbf{sort} sorts lines of text files:
	\item [\texttt{c}] checks for sorted input,
	\item [\texttt{f}] folds lower case to upper case characters,
	\item [\texttt{g}] compares general numerical values,
	\item [\texttt{h}] compares human readable numbers,
	\item [\texttt{k}] sorts via a key,
	\item [\texttt{n}] compares string numerical values,
	\item [\texttt{r}] reverses the results,
	\item [\texttt{s}] stabilizes the sort.
\end{enumx}

\begin{enumx}
	\item [\cmd] \textbf{tr} translates or deletes characters:
	% \item \texttt{tr abc xyz} changes \texttt{a} to \texttt{x}, $\ldots$,
	\item [c] uses the complement of ,,set1'',
	\item [d] deletes characters, does not translate,
	\item [s] replaces each sequence of a repeated character that is listed 
	in the last specified ,,set'' with a single occurrence of that character.
\end{enumx}

\begin{enumx}
	\item [\cmd] \textbf{uniq*} omits repeated lines:
	\item [\texttt{c}] prefixes lines by the number of occurences
	\item [\texttt{d}] only prints duplicate lines, one for each group
	\item [\texttt{f}] avoids comparing first fields
	\item [\texttt{i}] ignores differences in case
	\item [\texttt{s}] avoids comparing first characters
	\item [\texttt{w}] compares no more than $n$ characters
\end{enumx}

% \textbf{vim} is a programmers text editor.

\begin{enumx}
	\item [\cmd] \textbf{wc}* prints newline, word and byte counts (\texttt{lwc}):
	%\item [\texttt{c}] prints the byte counts,
	%\item [\texttt{l}] prints the newline counts,
	\item [\texttt{m}] prints the character counts,
	%\item [\texttt{w}] prints the word counts.
	\item [\texttt{L}] prints the maximum display width.
\end{enumx}

\begin{enumx}
	\item [\cmd] \textbf{xargs} builds and executes command lines from standard input.
\end{enumx}

\begin{enumx}
	\item [\cmd] \textbf{yes} outputs a string repeatedly until killed.
\end{enumx}


\subsection{Shell builtins}
\begin{enumx}
	\item [\cmd] \textbf{alias} allows a string to be substituted for a word.
	\item [\cmd] \textbf{cd} changes the shell working directory:
	\item [\texttt{-}] to the previous directory.
	\item [\cmd] \textbf{echo}* displays a line of text:
	\item [\texttt{e}] enables interpretation of backslash escapes,
	\item [\texttt{n}] does not output the trailing newline.
	\item [\cmd] \textbf{test} checks file types and compares values.
	\item [\cmd] \textbf{unset} unsets a shell variable, removing it from memory and the shell's exported environment.
	\item [\cmd] \textbf{wait} waits for process to change state.
\end{enumx}


\subsection{Networking}
\begin{enumx}
\item [\cmd] \textbf{curl} transfers a URL.
\item [\cmd] \textbf{dig} interrogates DNS name servers.                        
\item [\texttt{x}] performs a simplified reverse lookup. 
\item [\cmd] \textbf{host} is a DNS lookup utility.  
\item [\cmd] \textbf{ifconfig} configures a network interface.   
\item [\cmd] \textbf{inetd} is a super-server daemon that provides Internet services.
\item [\cmd] \textbf{netcat}: arbitrary TCP and UDP connections and listens.
\item [\cmd] \textbf{netstat} prints network connections, routing tables, 
interface statistics, masquerade connections, and multicast memberships.
\item [\cmd] \textbf{nslookup} queries Internet name servers interactively.
\item [\cmd] \textbf{ping} tests the reachability of a host 
on an IP network by sending ICMP ECHO\_REQUEST:
\item [\texttt{c}] stops after sending ,,count'' packets,
\item [\texttt{n}] numeric output only, 
	avoids to lookup symbolic names for host addresses. 
\item [\cmd] \textbf{rdate} sets the system's date from a remote host.
\item [\cmd] \textbf{rlogin} starts a terminal session on a remote host.
\item [\cmd] \textbf{route} shows and manipulates the IP routing table.
\item [\cmd] \textbf{ssh} is an OpenSSH SSH client (remote login program).
\item [\texttt{D}] (bind address)
\item [\texttt{p}] (port)
\item [\texttt{X}] (X11 forwarding)
\item [\cmd] \textbf{traceroute} is a computer network diagnostic tool for 
displaying the route (path) and measuring transit delays of 
\item [\cmd] \textbf{wget} is a non-interactive network downloader.
\item [\texttt{A}, \texttt{R}] specifies lists 	of file suffixes or 
	patterns (when wildcard characters appear) to accept or reject,
\item [\texttt{b}] goes to background immediately after startup,
\item [\texttt{c}] continues getting a partially-downloaded file,
\item [\texttt{m}] turns on options suitable for mirroring: 
	infinite recursion and time-stamping,
\item [\texttt{np}] does not ever ascend to the
	parent directory when retrieving recursively,
\item [\texttt{U}] identifies as ,,agent-string'' to the HTTP server.
\item [\texttt{w}] waits the specified number of seconds 
	between the retrievals (see also \texttt{--random-wait}).
\end{enumx}


\subsection{Searching}
\begin{enumx}
\item [\cmd] \textbf{find} searches for files in a directory hierarchy.
\item [\cmd] \textbf{locate} finds files by names.
\item [\cmd] \textbf{updatedb} updates the file database used by locate.
\item [\cmd] \textbf{whatis} displays one-line manual page description.
\item [\cmd] \textbf{whereis} locates the binary, source, 
and manual page files for a command.
\end{enumx}


%\subsection{Documentation}

\subsection{Miscellaneous}
\begin{enumx}
\item [\cmd] \textbf{bc} is an arbitrary precision calculator language.
\item \texttt{echo 'obase=16;255' | bc} prints \texttt{FF},
\item \texttt{echo 'ibase=2;obase=A;10' | bc} prints \texttt{2},
\item \texttt{scale=10} (after \texttt{bc -l}) sets working precision.
\item [\cmd] \textbf{dc} is a reverse-polish desk calculator.
One of the oldest Unix utilities, 
predating even the invention of the C programming language.
\item [\cmd] \textbf{cal}, \textbf{ncal} displays a calendar.
\item [\texttt{e}] displays date of Easter,
\item [\texttt{j}] displays Julian days,
\item [\texttt{m}] displays the specified month,
\item [\texttt{w}] prints the numbers of the weeks,
\item [\texttt{y}] displays a calendar for the specified year,
\item [\texttt{3}] displays the previous, current and next month.
\item [\cmd] \textbf{date} prints or set the system date and time.
% \textbf{expr}
%\item [\cmd] \textbf{lp} prints files.
%\item [\cmd] \textbf{od} dumps files in octal.
% hexdump -C, xxd
\item [\cmdblack] \textbf{seq} prints a sequence of numbers:
\item [\texttt{w}] equalizes width by padding with leading zeroes.
\item [\cmdblack] \textbf{sleep} delays for a specified amount of time.
\item [\cmd] \textbf{true}, \textbf{false} does nothing, (un)successfully.
\end{enumx}

\end{multicols}
\end{document}

Todo:
\textbf{gp} invokes the PARI/GP calculator.
\textbf{pdflatex} runs the pdfTeX typesetter.

\textbf{apropos} searches the manual page names and descriptions.
\textbf{man} is an interface to the online reference manuals.

\textbf{ghci} is the Glasgow Haskell Compiler.
\textbf{ipython} is an interactive Python shell, see also
\textbf{python} and \textbf{python3}.
\textbf{gcc} is a C and C++ compiler.
