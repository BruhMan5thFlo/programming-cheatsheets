\renewcommand\theFancyVerbLine{\normalsize\arabic{FancyVerbLine}}

\begin{enumx}
    \item [\cmd] \textbf{awk} is a pattern scanning / processing language,
    a pseudo-C interpretor.
    Sample code:
\begin{minted}[linenos, numbersep=3pt, frame=lines, framesep=1mm]{bash}
BEGIN {print "- Start -"}
/word/ {print NR ")" $1, $2}
END {print "- End -"}
\end{minted}

\item [] Examples of conditions:
\begin{enumx}
    \item \texttt{/word[0+9]+/}: regular expressions
    \item \texttt{!/word[0+9]+/}: regexes inverted
    \item \texttt{$\sim$} and \texttt{!$\sim$}: matches / does not match.
    \item \texttt{length(\$0) > 18}.
\end{enumx} 

\item [] Important variables:
\begin{enumx}
    \item FS: field separator (tab),
    \item OFS: output field separator,
    \item RS: record separator (new line),
    \item NR: number of the current record,
    \item NF: number of fields in the current record.
\end{enumx} 

\item [\cmd] \textbf{grep} prints lines matching a pattern:
\item [\texttt{c}] prints a count of matching lines instead,
\item [\texttt{e}] uses a ,,regexp'' pattern,
\item [\texttt{f}] obtains patterns from a file,
\item [\texttt{i}] ignores case disctinctions,
\item [\texttt{v}] inverts the sense of matching,
\item [\texttt{w}] selects only lines containing matches that form whole words,
\item [\texttt{n}] prints line numbers as well,
\item [\texttt{A}] prints ,,num'' lines of trailing content,
\item [\texttt{B}] prints ,,num'' lines of leading content,
\item [\texttt{C}] prints ,,num'' lines of both contents,
\item [\texttt{R}] ???,
\item [\cmd] \textbf{sed}: a stream editor filtering/transforming text.
\end{enumx}

\begin{enumx}
	\item [\cmd] \textbf{comm} compares two sorted files line by line.
\end{enumx}

\begin{enumx}
	\item [\cmdblack] \textbf{cut} prints selected parts of lines:
	\item [] \texttt{-}\texttt{-}\texttt{complement} complements the selection,
	\item [\texttt{c}] selects only these characters,
	\item [\texttt{d}] uses ,,delim'' instead of Tab for field delimeter,
	\item [\texttt{f}] selects only these fields,
	\item [\texttt{s}] does not print lines not containing delimeters.
	\item [\cmdblack] \textbf{join} joins lines of two files on a common field.
	\item [\cmdblack] \textbf{paste} merges lines of files.
	\item [\texttt{d}] reuses characters from ,,list'' instead of tabs,
	\item [\texttt{s}] pastes one file at a time, not in parallel.
\end{enumx}

\begin{enumx}
	\item [\cmd] \textbf{diff} compares files line by line:
	\item [\texttt{y}] outputs in two columns,
	\item [\texttt{i}] ignores case differences,
	\item [\texttt{w}] ignores all white space.
	% E Z b B
\end{enumx}

\begin{enumx}
	\item [\cmd] \textbf{fmt} is a simple optimal text formatter, 
	\item [\cmd] \textbf{fold} wraps each line to fit in specified width.
\end{enumx}

\begin{enumx}
	\item [\cmd] \textbf{head}* outputs the first (last) part of files:
	\item [\texttt{c}] the first ,,num'' bytes,
	\item [\texttt{n}] the first ,,num'' lines,
	\item [\cmd] \textbf{tail*} the last ,,num'' bytes:
	\item [\texttt{c}] the last ,,num'' bytes,
	\item [\texttt{n}] the last ,,num'' lines,
	\item [\texttt{f}] outputs appended data as the file grows,
	\item [\texttt{s}] sleeps for approximately ,,n'' seconds between iteration 
\end{enumx}

\begin{enumx}
	\item [\cmd] \textbf{less} is opposite of \textbf{more}.
	\item [\cmd] \textbf{more} is a file perusal filter for crt viewing.
\end{enumx}

\begin{enumx}
	\item [\cmd] \textbf{nl*} numbers lines of files:
	\item [\texttt{s}] adds ,,string'' after line number,
	\item [\texttt{w}] uses ,,number'' columns for line numbers.
\end{enumx}

\begin{enumx}
	\item [\cmd] \textbf{shuf*} generates random permutations:
	\item [\texttt{e}] treats each ,,arg'' as an input line,
	\item [\texttt{i}] treats each number .. through .. as an input line, 
	\item [\texttt{n}] outputs at most ,,count'' lines,
	\item [\texttt{r}] output lines can be repeated (with \texttt{-n}).
\end{enumx}

\begin{enumx}
	\item [\cmdblack] \textbf{sort} sorts lines of text files:
	\item [\texttt{c}] checks for sorted input,
	\item [\texttt{f}] folds lower case to upper case characters,
	\item [\texttt{g}] compares general numerical values,
	\item [\texttt{h}] compares human readable numbers,
	\item [\texttt{k}] sorts via a key,
	\item [\texttt{n}] compares string numerical values,
	\item [\texttt{r}] reverses the results,
	\item [\texttt{s}] stabilizes the sort.
\end{enumx}

\begin{enumx}
	\item [\cmd] \textbf{tr} translates or deletes characters:
	% \item \texttt{tr abc xyz} changes \texttt{a} to \texttt{x}, $\ldots$,
	\item [c] uses the complement of ,,set1'',
	\item [d] deletes characters, does not translate,
	\item [s] replaces each sequence of a repeated character that is listed 
	in the last specified ,,set'' with a single occurrence of that character.
\end{enumx}

\begin{enumx}
	\item [\cmd] \textbf{uniq*} omits repeated lines:
	\item [\texttt{c}] prefixes lines by the number of occurences
	\item [\texttt{d}] only prints duplicate lines, one for each group
	\item [\texttt{f}] avoids comparing first fields
	\item [\texttt{i}] ignores differences in case
	\item [\texttt{s}] avoids comparing first characters
	\item [\texttt{w}] compares no more than $n$ characters
\end{enumx}

% \textbf{vim} is a programmers text editor.

\begin{enumx}
	\item [\cmd] \textbf{wc}* prints newline, word and byte counts (\texttt{lwc}):
	%\item [\texttt{c}] prints the byte counts,
	%\item [\texttt{l}] prints the newline counts,
	\item [\texttt{m}] prints the character counts,
	%\item [\texttt{w}] prints the word counts.
	\item [\texttt{L}] prints the maximum display width.
\end{enumx}

\begin{enumx}
	\item [\cmd] \textbf{xargs} builds and executes command lines from standard input.
\end{enumx}

\begin{enumx}
	\item [\cmd] \textbf{yes} outputs a string repeatedly until killed.
\end{enumx}
