\section{Programming in Bash}
\subsection{Shebang}
The shebang (\texttt{\#!}) at the head of a script indicates an 
interpreter for execution, as in \texttt{\#!/bin/bash}.
Lines starting with a \texttt{\#} (with the exception of shebang) 
are comments and thus won't be executed.

There are always three default files open:
\emph{stdin} (the keyboard, file descriptor 0),
\emph{stdout} (the screen, file descriptor 1) and
\emph{stderr} (error messages output to the screen, file descriptor 2).

These \textbf{streams} can be \textbf{redirected}: 
\texttt{cmd > file} redirects to a file (overwrites),
\texttt{cmd >{}> file} appends instead,
\texttt{m>n} (or \texttt{m>\&n}) redirects a file descriptor to a file 
(or another file descriptor), 
\texttt{\&>file} redirects both stdout and stderr to a file;
\texttt{:> file} truncates file to zero length and
\texttt{|} (pipe) serves as a command chaining tool.
