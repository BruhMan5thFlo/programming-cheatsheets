\section{Programming in Bash}
\subsection{Shebang}
The shebang (\texttt{\#!}) at the head of a script indicates an 
interpreter for execution, as in \texttt{\#!/bin/bash}.
Lines starting with a \texttt{\#} (with the exception of shebang) 
are comments and thus won't be executed.

There are always three default files open:
\emph{stdin} (the keyboard, file descriptor 0),
\emph{stdout} (the screen, file descriptor 1) and
\emph{stderr} (error messages output to the screen, file descriptor 2).

These \textbf{streams} can be \textbf{redirected}: 
\texttt{cmd > file} redirects to a file (overwrites),
\texttt{cmd >{}> file} appends instead,
\texttt{m>n} (or \texttt{m>\&n}) redirects a file descriptor to a file 
(or another file descriptor), 
\texttt{\&>file} redirects both stdout and stderr to a file;
\texttt{:> file} truncates file to zero length and
\texttt{|} (pipe) serves as a command chaining tool.

\subsection{Variables}
\textbf{Variables} are case sensitive and capitalized by default.
Variables can also contain digits and underscores,
but a name starting with a digit is not allowed.
Example: \texttt{var=value; echo \${var}s} prints \texttt{values}.
Special variables:
\begin{enumx}
	\item \texttt{\$0}, \texttt{\$1}, \ldots: name of the script itself,
	the first, second, etc. argument.
	\item \texttt{\$*} and \texttt{\$@} denote all the positional parameters.
	\item \texttt{\$\#}: the number of positional parameters
	\item \texttt{\$?}: exit status of the most recently executed command.
	\item \texttt{\$\$}: the process ID of the shell.
	\item \texttt{\$!}: the process ID of the most recently executed command.
\end{enumx}

% special meaning of < > ; | * ? - one has to escape them
% ' '
% " "
