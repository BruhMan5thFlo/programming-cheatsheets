\begin{enumx}
	\item [\cmd] \textbf{clear} clears the terminal screen.
	\item [\cmd] \textbf{env} runs a program in a modified environment.
	\item [\cmd] \textbf{exit} terminates the calling process.
	\item [\cmd] \textbf{finger} looks up user information.
	\item [\cmd] \textbf{history}  displays the history list. %with line numbers.
%	\item [\cmd] \textbf{logname} prints user's login name.
	\item [\cmd] \textbf{mesg} displays messages from other users.
\end{enumx}

\begin{enumx}
	\item [\cmd] \textbf{passwd} changes user password:
	\item [\texttt{d}] deletes an account's password (makes it empty),
	\item [\texttt{e}] expires an account's password,
	\item [\texttt{n}] sets minimum days to change password,
	\item [\texttt{w}] sets warning days before password expire,
	\item [\texttt{x}] sets the maximum number of days a password remains valid.
\end{enumx}

\begin{enumx}
	\item [\cmd] \textbf{su} changes user ID or becomes superuser.
	\item [\cmd] \textbf{sudo} executes a command as another user.
\end{enumx}

\begin{enumx}
	\item [\cmd] \textbf{uname} prints system information:
	\item [\texttt{a}] all information, in the following order:
	\item [\texttt{s}] the kernel name,
	\item [\texttt{n}] the network node hostname,
	\item [\texttt{r}] the kernel release,
	\item [\texttt{v}] the kernel version,
	\item [\texttt{m}] the machine hardware name,
	\item [\texttt{p}] the processor type,
	\item [\texttt{i}] the hardware platform,
	\item [\texttt{o}] the operating system.
\end{enumx}

\begin{enumx}
	\item [\cmd] \textbf{uptime}: how long has the system been running?
\end{enumx}

\begin{enumx}
	\item [\cmd] \textbf{wall} writes a message to all users,
	\item [\cmd] \textbf{write} sends a message to another user. 
\end{enumx}

\begin{enumx}
	\item [\cmd] \textbf{who} shows who is logged on,
	\item [\cmd] \textbf{w} does the same and shows what they are doing,
	\item [\cmd] \textbf{whoami} prints effective userid.
\end{enumx}

